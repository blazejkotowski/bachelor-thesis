\chapter{Solution}
\label{chap:solution}

\section{Overview}

The proposed solution is to transmit payload data using nine sinusoidal signals hidden in host signal (eight carrier signals and one synchronization signal). Two algorithms are required: one for sound synthesis and one for recording analysis.

The synthesis algorithm:
\begin{enumerate}
\item splits payload data into frames
\item uses frame bytes to modulate carrier signals, utilizing DBPSK~[\ref{subsec:dbpsk}]
\item adjusts amplitude of data signal using a psychoacoustic model
\item filters host signal, removing frequencies around data signal
\item adds data signal to filtered host signal
\end{enumerate}

The analysis algorithm:
\begin{enumerate}
\item demodulates the recorded signal into a stream of bytes
\item uses a frame detection algorithm to extract frame data
\item for each frame, applies error correction and outputs the result bytes
\end{enumerate}

\begin{figure}[h]
\centering
\begin{tikzpicture}[auto,node distance=1cm,>=stealth']
\tikzset{
block/.style= {draw, rectangle, minimum height=2em,minimum width=4em},
textblock/.style= {minimum height=2em,minimum width=4em},
sum/.style = {draw, circle},
oscillator/.style = {draw, circle},
input/.style  = {coordinate},
output/.style = {coordinate}
}

\node [textblock](payload) {payload};
\node [block, right = 1cm of payload](framing) {encoder};
\node [block, below = 1cm of framing, minimum height = 18em](modulator) {modulator};

\node [oscillator, left = 1cm of modulator, align = center](oscillator4) {$\sim$};
\node [oscillator, above = 0.1cm of oscillator4](oscillator3) {$\sim$};
\node [oscillator, above = 0.1cm of oscillator3](oscillator2) {$\sim$};
\node [oscillator, above = 0.1cm of oscillator2](oscillator1) {$\sim$};
\node [oscillator, above = 0.1cm of oscillator1](oscillator0) {$\sim$};
\node [oscillator, below = 0.1cm of oscillator4](oscillator5) {$\sim$};
\node [oscillator, below = 0.1cm of oscillator5](oscillator6) {$\sim$};
\node [oscillator, below = 0.1cm of oscillator6](oscillator7) {$\sim$};
\node [oscillator, below = 0.1cm of oscillator7](oscillator8) {$\sim$};

\node [textblock, below = 1cm of oscillator8](host) {host};
\node [block, right = 3.25cm of host](filter) {filter};
\node [block, above = 1cm of filter, text centered, text width = 2.5cm](psychmodel) {psychoacoustic model};
\node [sum, right = 1cm of modulator](applymodel) {$\times$};
\node [sum, right = 1cm of applymodel](sumwithhost) {$+$};
\node [block, right = 3cm of sumwithhost](demodulator) {demodulator};
\node [block, below = 1cm of demodulator](framerecovery) {frame recovery};
\node [textblock, below = 1cm of framerecovery](outputpayload) {payload};

\draw [->] (payload) -- (framing);
\draw [->] (framing) -- (modulator);
\draw [->] (oscillator0) -- (modulator.west|-oscillator0);
\draw [->] (oscillator1) -- (modulator.west|-oscillator1);
\draw [->] (oscillator2) -- (modulator.west|-oscillator2);
\draw [->] (oscillator3) -- (modulator.west|-oscillator3);
\draw [->] (oscillator4) -- (modulator.west|-oscillator4);
\draw [->] (oscillator5) -- (modulator.west|-oscillator5);
\draw [->] (oscillator6) -- (modulator.west|-oscillator6);
\draw [->] (oscillator7) -- (modulator.west|-oscillator7);
\draw [->] (oscillator8) -- (modulator.west|-oscillator8);
\draw [->] (modulator) -- (applymodel);

\draw [->] (host) -- (filter);
\draw [->] (filter) -- (psychmodel);
\draw [->] (filter) -| (sumwithhost);
\draw [->] (psychmodel.north-|applymodel) -| (applymodel);

\draw [->] (applymodel) -- (sumwithhost);

\draw [->] (sumwithhost) -- node {sound} (demodulator);

\draw [->] (demodulator) -- (framerecovery);
\draw [->] (framerecovery) -- (outputpayload);
\end{tikzpicture}
\caption{Block diagram of proposed solution}
\label{fig:solution-diagram}
\end{figure}

\section{Synthesis algorithm}

The signal synthesis algorithm has some configurable parameters:

\begin{itemize}
\item d -- duration of signal which encodes single byte (seconds)
\item fs -- sampling rate
\item sync -- synchronization signal frequency (Hertz)
\item carriers -- array of carrier signals frequencies (in Hertz)
\end{itemize}

\subsection{Data frame}

To ensure robust payload decoding in spite of transmission errors, data is encoded in frames consisting of 24 bytes:

\begin{itemize}
\item preamble -- 2 bytes
\item payload -- 16 bytes
\item error correction data -- 6 bytes
\end{itemize}

% TODO: frame visualization
A preamble is used in frame recovery algorithm~[\ref{sec:frame-recovery}] to detect frame beginning. It consists of 2 ASCII synchronous idle characters (\texttt{0x16}).
\texttt{0x} notation denotes a byte in hexadecimal format.

Error correction data is calculated from payload using Reed-Solomon~[\ref{subsec:reedsolomon}]~code.

\subsection{Carrier modulation}

Single carrier signal is a sine wave, modulated by a stream of bits using differential binary phase shift keying~[\ref{subsec:dbpsk}].

DBPSK has the advantages over other modulation methods for audio domain:~\cite{Applidium}

\begin{itemize}
\item more robust and resilient to noise than amplitude modulation
\item uses less bandwidth than frequency modulation, meaning narrower band will be removed from host signal
\item does not require reference signal in demodulator
\end{itemize}

However, sudden phase transitions required by DBPSK cause audible parasitic effects ("clicking"). To mitigate that, the parts of the signal at transition time are multiplied by cosine window.

\begin{figure}[h]
  \includegraphics[width=\linewidth]{figures/phase-transition}
  \caption{Sudden phase transition of $\pi$}
  \label{fig:phase-transition}
\end{figure}

\begin{figure}[h]
  \includegraphics[width=\linewidth]{figures/smooth-phase-transition}
  \caption{Smooth phase transition of $\pi$}
  \label{fig:smooth-phase-transition}
\end{figure}

\subsection{Byte transmission}

Total of nine sine signals are transmitted simultaneously. Eight of them each encode a single bit of a byte. The remaining signal is a synchronization signal, modulated by stream of binary ones, is used for detecting byte start\,/\,end.

% TODO: figure of synchronization signal

\subsection{Insetting in host signal}

Insetting carrier signal in host signal consists of 4 steps:

\begin{enumerate}
\item host signal is filtered with bandstop filter, removing frequencies around carrier signal
\item for each window, masking threshold for carrier signal is calculated using psychoacoustic model described in~[\ref{subsec:psychoacoustic-model}]
\item amplitude of carrier signal is multiplied by value under masking threshold
\item carrier signal is added to filtered host signal
\end{enumerate}

\section{Analysis algorithm}

Analysis algorithm shares configuration parameters with synthesis algorithm, and can further be parameterized with:

\begin{itemize}
\item ws -- window size for analysis (samples)
\end{itemize}

Each ws consecutive recorded samples are analyzed by analysis algorithm.

\subsection{Carrier demodulation}

A phase reading is obtained for each of the carrier frequencies and synchronization frequency.

For each frequency:
\begin{enumerate}
\item recorded frame is filtered with a bandpass filter centered at that frequency and multiplied by Hanning window % TODO: ref
\item phase shift relative to beginning of frame is calculated using Goertzel algorithm~[\ref{subsec:goertzel}]
\item absolute phase shift ($|\varphi|$) is obtained by calculating phase offset ($\delta\varphi$) of current frame and subtracting it from relative phase ($\varphi$).
\end{enumerate}

\begin{equation}
\delta\varphi = 2\pi(\frac{\textrm{framestart} \cdot \textrm{frequency}}{\textrm{fs}} \pmod{1.0})
\end{equation}

Where framestart is the index of the first sample of a frame in a stream of all recorded samples, and fs is sampling rate.

\begin{equation}
|\varphi| = \varphi - \delta\varphi
\end{equation}

When the synchronization signal has stable phase for the predefined number of readings (rpb -- readings per byte), it signals a byte transmission.
Phase is considered stable when standard deviation of rpb previous readings is below $\pi/5$.

\begin{equation}
\textrm{rpb} = \lfloor\frac{\textrm{fs} \cdot \textrm{d}}{\textrm{ws}}\rfloor
\end{equation}

When byte transmission occurs, previous rpb bit carrier phase readings are averaged and saved. Then the averages are compared to previous averages and if difference of $\pi$ is detected, a binary 1 is decoded. Otherwise, binary 0 is decoded.

Eight bits decoded from carrier signals are then combined to form a single byte, which gets forwarded to frame recovery algorithm.

\subsection{Frame recovery}
\label{sec:frame-recovery}

A simple frame recovery algorithm scans the byte stream for preamble bytes.
However, since transmission error might have occurred and preamble might be scrambled, the algorithm has to evaluate each consecutive pair of bytes as a candidate for being a preamble.

It assigns a score for each position in bit stream, calculated as follows:

\begin{equation}
\textrm{score} = 1 - \frac{\textrm{pop(bit0 xor \texttt{0x16})} + \textrm{pop(bit1 xor \texttt{0x16})}}{16}
\end{equation}

(where pop is a function returning number of set bits in a byte).

Score of 1 means exact preamble is detected. However, if no preamble position is found, the algorithm waits for next 24 bytes to calculate the score of a position a frame size bytes ahead. If that score is large enough, previous position is selected as frame beginning.

Once beginning of frame is identified and all frame bytes are received, the algorithm extracts payload data, tries to apply Reed-Solomon error correction to correct any transmission errors, and returns decoded part of payload data.
