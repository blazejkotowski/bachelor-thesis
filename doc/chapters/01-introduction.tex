\chapter{Introduction}

\section{Motivation}

Majority of communications in modern age are carried through electromagnetic waves. However, this is not the only possible medium of communication.
The ubiquity of cheap audio equipment, like loudspeakers and microphones, creates a possibility for data transmission via sound.

Unlike electromagnetic waves, a range of sound pressure waves can be perceived by humans. This creates an issue when one does not want listeners to be aware that data is transmitted.
A way of solving this problem is insetting the data signal into a host sound in a specially crafted manner, exploiting the properties of human auditory system.
The host sound should mask the data signal, making it imperceptible to humans, while still detectable by machines.

As sound production, transmission and recording is commonplace, such system would have many applications, some of which are discussed in Chapter \ref{chap:summary}.

\section{Thesis Objective}

This work aims at achieving the following goals:

\begin{itemize}
  \item implementation of software system for insetting and recovery of data in host audio signal
  \item evaluation of implemented system with regard to criteria of transmission robustness and distortion imperceptibility.
\end{itemize}

The system should take existing audio file, like a piece of music, and data to be transmitted, and combine them to generate a sound to be played by loudspeakers.
This sound should be as close to original audio as possible, i.e. the human listener should not be able to perceive the distortion introduced by hiding the payload data.
On the receiving side, another program should be able to recover the payload from recorded sound.

\section{Thesis Outline}

\begin{itemize}
  \item Chapter 2 gives a brief overview of the theoretical concepts required for the solution, focusing on digital signal processing concepts as well as communications and coding techniques
  \item Chapter 3 describes the proposed solution of the problem
  \item In Chapter 4, implementation details of the system are presented, including relevant excerpts of python source code
  \item Chapter 5 presents results of experimental evaluations, outlining system's performance in various conditions
  \item Finally, Chapter 6 sums up the work, including possible applications of the system, as well as identifying possible areas for improvement.
\end{itemize}

For purposes of this thesis, Błażej Kotowski is responsible for implementation and experimental evaluation. Tomasz Pewiński is responsible for designing a solution and implementation.
