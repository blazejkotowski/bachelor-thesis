\chapter{Summary}
\label{chap:summary}

The thesis presents a solution for insetting and recovery of data in sound in imperceptible way, as well as
implementation of the solution.

The solution is shown to allow a robust transmission in presence of interference (in laboratory conditions) and to be
imperceptible to human listeners (See \ref{sec:psychoacoustic-test}).

\section{Possible applications}

The python library developed as a part of this thesis can be utilized on many platforms that support python programming language.

Example applications might include:

\begin{enumerate}
\item In marketing -- embedding data in advertisements.

Data like product details or promotional coupons could be hidden in advertisement music. Consumers would be able to
extract and use this data using smartphone application.

\item In broadcasting -- embedding meta-data in music.

Meta-data like artist name, song title or even song lyrics could be hidden in music broadcasted on live events.
Listeners would be able to access it using mobile application.

\item In digital rights management -- a simple watermark.

Watermark data could be hidden in music or film score to detect copyright infringement. The disadvantage of presented
solution is ease of watermark removal.
\end{enumerate}

\section{Possible improvements}

\begin{itemize}
\item Increase data bandwidth

Transmission bandwidth achieved in the implemented solution might limit the number of applications.
Bandwidth could be increased by decreasing single byte transmission duration and using phase recovery method
less susceptible to noise.

\item Improve frame recovery algorithm

Implemented frame recovery algorithm is shown to fail to correctly detect a frame in some conditions (see~\ref{subsub:frame-detection}).
This could be improved by using longer preamble or another method for synchronizing frames.

\item Evaluation in real-life conditions

The tests in Chapter \ref{chap:tests} are performed in laboratory conditions. Additional tests should be
performed to evaluate the system's performance in presence of natural phenomena (like echoes) and using commodity hardware.

\item Advanced psychoacoustic model

Another psychoacoustic model might be implemented, exploiting additional properties of human auditory system, like
temporal masking~(\ref{itm:masking-effects}).

\item Implementation of mobile application

A showcase application for easy broadcasting and receiving of data, running on any of major mobile operating systems.
\end{itemize}

\section{Acknowledgments}

Authors would like to thank prof.~dr. hab.~inż~Adam Dąbrowski and dr. Szymon Drgas for support in designing a solution,
and dr.~inż. Andrzej Meyer for help with laboratory and audio equipment.
